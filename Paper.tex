%%%%%%%%%%%%%%%%%%%%%%%%%%%%%%%%%%%%%%%%%%%%%%%%%%
%-----% CLASSIC STEM PAPER TEMPLATE %------------%
%%%%%%%%%%%%%%%%%%%%%%%%%%%%%%%%%%%%%%%%%%%%%%%%%%

%%%%%%%%%%%%%%%%%%%%%%%%%%%%%%%%%%%%%%%%%%%%%%%%%%
%-----% SOURCE CODE LICENSE %--------------------%
%%%%%%%%%%%%%%%%%%%%%%%%%%%%%%%%%%%%%%%%%%%%%%%%%%
%%% This template was drafted by Ethan Barry, but
%%% has been created with tidbits from many diff-
%%% erent StackExchange discussions. :)
%%% You may have it under the terms
%%% of the BSD 2-clause license. Go wild.
%%% -----
%%% THIS IS NOT OFFICIALLY ENDORSED BY THE UNIV-
%%% ERSITY OF TEXAS AT TYLER, OR ITS FACULTY.
%%%%%%%%%%%%%%%%%%%%%%%%%%%%%%%%%%%%%%%%%%%%%%%%%%

%%% Frontmatter inspired by Mario Vilar's great
%%% template at:
%%%     github.com/mariovilar/Thesis-Template
%%% No code from that document was used, however.

%%%%%%% PRELUDE %%%%%%%%%%%%%%%%%%%%%%%%%%%%%%%%%%
\documentclass[letterpaper,
               extrafontsizes,
               11pt,
               oneside,
               final]{memoir}   % The superior LaTeX class.

\usepackage{amsmath}
\usepackage{amsfonts}
\usepackage{amssymb}
\usepackage{amsthm}             % memoir does not provide theorem environments.
\usepackage[english]{babel}     % Split words correctly.
\usepackage{caption}
                                % To change the font, alter the package import below.
                                % But don't, unless you want a typographic abomination.
                                % NOTE: If you change the font, it is very likely you
                                %       will have to hunt down and remove the uses of \textsw
                                %       and \swshape in this file, as many fonts do not provide
                                %       swash italics. You've been warned!
\usepackage{ebgaramond}[RawFeature={+hlig,+hist}]
\usepackage{ebgaramond-maths}   % Make your math font awesome, too.
\usepackage{fontenc}
\usepackage{graphicx}
\usepackage[hidelinks]{hyperref}
\usepackage{microtype}          % Makes adjustments to spacing.
\usepackage{pgfpages}
\usepackage{quiver}             % Use <https://q.uiver.app/> to draw tikz diagrams!
\usepackage{tikz}
                                % The package below is needed by ebgaramond-maths.
\usepackage[math-style=ISO,
            bold-style=ISO]{unicode-math}
\usepackage{witharrows}         % Draw arrows between equations. Very handy.
\usepackage{xcolor}
\usepackage{xfrac}              % Proper 'slanted' fractions.

\usepackage{lipsum}             % Remove this when not in use.

\WithArrowsOptions{displaystyle}
\setmathfont{Garamond-Math.otf}[StylisticSet={2,7,9}] % Awesome math font. See documentation.

%%%%%%%%%%%%%%%%%%%%%%%%%%%%%%%%%%%%%%%%%%%%%%%%%%
%-----% GLOBAL VARIABLES %-----------------------%

%%% Set these to your desired values if you want
%%% to use the titlepage.
\newcommand{\thedocumenttitle}{Document Title}
\newcommand{\thedepartment}{Departments of Mathematics \& Computer Science}
\newcommand{\theuniversity}{University of Texas at Tyler}
\newcommand{\thepublisher}{UT Tyler Press}
\newcommand{\theamsclassif}{00000, 10101}
\newcommand{\thecity}{Tyler, Texas}
\newcommand{\thecountry}{The United States of America}
\newcommand{\theauthors}{Author I, Author II, Author III, Author IV}

%%%%%%%%%%%%%%%%%%%%%%%%%%%%%%%%%%%%%%%%%%%%%%%%%%
%-----% PAGE LAYOUT %----------------------------%

%%% The memoir class makes headings a breeze.
%%% Just do
%%% \makeXhead{<style>}{<left>}{<center>}{<right>}
%%% to set up the heading style.
\makepagestyle{myheadings}
\makeevenhead{myheadings}{\thepage}{\scshape{\MakeTextLowercase\thedocumenttitle}}{}
\makeoddhead{myheadings}{}{\scshape{chapter~\thechapter}}{\thepage}
\pagestyle{myheadings}

%-----% MARGINS & TYPEBLOCK %--------------------%

%%% This template includes a large right margin
%%% for margin notes. Margin notes should be used
%%% to introduce variables, terms, &c.
\setlrmarginsandblock{0.85in}{1.65in}{*}    % Left-right margins.
\setulmarginsandblock{0.85in}{1in}{*}       % Upper-lower margins.
\setmarginnotes{2em}{7em}{2em}              % Better spacing for margin notes on letterpaper.
\setheadfoot{0.2in}{0.4in}
\checkandfixthelayout[lines]                % THIS MUST BE CALLED.
\OnehalfSpacing                             % Like double-spacing, but better.

%-----% HEADINGS & SECTIONS %--------------------%
\chapterstyle{thatcher}                     % Multiple choices available. This one fits the template well.
\setbeforesecskip{3ex}                      % Set document division appearances.
\setbeforesubsecskip{3ex}
\setsecheadstyle{\raggedright\Large\scshape}
\setsubsecheadstyle{\raggedright\large\scshape}
\setsubsubsecheadstyle{\raggedright\large\swshape}

\maxsecnumdepth{subsubsection}              % Everything down to subsubsections will be numbered.
\setsecnumdepth{subsubsection}
%\renewcommand*{\thesection}{\arabic{section}} % Do not display chapter numbers. Useful in a paper.

%-----% REFERENCES & NAVIGATION %----------------%
\setcounter{tocdepth}{2}                    % What division level appears in the ToC. 2 means subsection.

%-----% MATH FORMATTING %------------------------%

%%% NOTE: If you don't like the short math env.
%%%       names, this is the place to change them!
\theoremstyle{plain}                        % See amsthm for details on this section.
\newtheorem{thm}{Theorem}
\newtheorem{lem}{Lemma}
\newtheorem{prp}{Proposition}[section]

\theoremstyle{definition}
\newtheorem{dfn}{Definition}[section]

\theoremstyle{remark}
\newtheorem{rem}{Remark}
\newtheorem{nte}{Note}

%-----% DOCUMENT MACROS %------------------------%

%%% MarginNote places a correctly-formatted note in the margin.
%%% Just pass the text.
\newcommand{\MarginNote}[1]{
  \marginpar{\footnotesize\itshape\raggedright #1}
}

%%% Places a variable declaration in an inline math environment.
%%% Use \quad or \> to space them.
\newcommand{\MarginVariable}[1]{{\footnotesize\marginpar{\( #1 \)}}}


%%%%%%%%%%%%%%%%%%%%%%%%%%%%%%%%%%%%%%%%%%%%%%%%%%
%-----% THE DOCUMENT %---------------------------%
\begin{document}

%-----% THE TITLE PAGE %-------------------------%
\setlrmarginsandblock{1in}{1in}{*}          % Center the title page material.
\checkandfixthelayout                       % Must call this.
\titlingpageend{\clearpage}{\clearpage}     % Eliminate extra blank pages. NOTE: If you use a twoside
                                            %                                    layout, YOU MUST UNDO THIS!!!
\begin{titlingpage}
  \begin{center}
  \vspace*{1in}

  {\LARGE\itshape Subject}

  \vspace{0.25in}

  {\Huge\scshape\thedocumenttitle}

  \vspace{0.25in}
  \newcommand{\theoldtabcolsep}{\tabcolsep}
  \setlength{\tabcolsep}{0.25in}

  \begin{tabular}{cccc}
    {\scshape Author I} & {\scshape Author II} & {\scshape Author III} & {\scshape Author IV}\\
    {\verb|ID|} & {\verb|ID|} & {\verb|ID|} & {\verb|ID|}
  \end{tabular}
  \setlength{\tabcolsep}{\theoldtabcolsep}

  \vfill

  \includegraphics[width=2in]{UT_Tyler_Seal_Orange.png}         % Use the seal when you want a good titlepage.
 %\includegraphics[width=3.5in]{UT_Tyler_Primary_Logo.png}\\    % Use this logo if you like horrible graphic design,
                                                                % or your document isn't so formal as to require the
                                                                % seal.
  \vspace{1in}
  \end{center}
\end{titlingpage}
%-----% THE COPYRIGHT PAGE %---------------------%
\begin{titlingpage}
  \setlength{\parindent}{0in} % NOTE: This doesn't seem to propagate outside this environment.
                              %       If it causes problems, just use \noindent here.
  {\scshape\large\thedepartment}\\
  {\scshape\large\theuniversity}\\
  \vspace{0.25in}

  {\itshape Published by:} {\scshape\thepublisher}\\
  {\itshape Template by: } {\scshape Ethan Barry} {\itshape (after Mario Vilar)}\\ % TODO: Remove me (if you want)!
  {\itshape AMS Classification: } {\scshape\theamsclassif}\\
  \vfill

  {\thecity, on 18 June, 2024 }\\
  ©\>{\scshape\theauthors}\,---\,2024\\
  {\itshape All Rights Reserved}\\
  \vspace{0.10in}

  {\scshape printed in {\MakeLowercase{\thecountry}}}
\end{titlingpage}
\setlrmarginsandblock{0.85in}{1.65in}{*}    % Go back to the wide right margin layout.
\checkandfixthelayout                       % Must call this.
%-----% THE TABLE OF CONTENTS %------------------%
\begin{titlingpage}
  \tableofcontents
\end{titlingpage}

%%%%%%%%%%%%%%%%%%%%%%%%%%%%%%%%%%%%%%%%%%%%%%%%%%
%-----% ACKNOWLEDGEMENTS %-----------------------%
\chapter*{Acknowledgements}
\begin{center}
\parbox[c]{0.8\textwidth}{
  \small\lipsum[1]
}
\end{center}

%%%%%%%%%%%%%%%%%%%%%%%%%%%%%%%%%%%%%%%%%%%%%%%%%%
%-----% CHAPTER ONE %----------------------------%
 \chapter{The Beginning Chapter...}
 \epigraph{If all else fails, immortality can always be assured by spectacular error.}{\textit{This Is An Epigraph}\\\textsc{John Kenneth Galbraith}}
 Typeset this!\footnote{Here is a footnote.}
 \lipsum[2]
 \section{Here is a section division...}
 \lipsum[6]\footnote{And another footnote...}
 \begin{enumerate}
   \item \lipsum[7]
   \item \lipsum[8]
 \end{enumerate}

 \subsection{And here is a subsection...}
 \lipsum[9]\MarginNote{Here is a margin note which can float and drop down.}

 \subsubsection{And here is a sub-sub-section...}
 \lipsum[10]\MarginVariable{x\>z}                % Introduce your variables in the margins!
 \[  z \in \mathbb{Z} \quad\text{\textit{\&}}\quad x \in \mathbb{R} \]
 \[ \int_a^b f'(t)\,dt = f(b) - f(a) \]
 \begin{thm}
  This is some theoremic text where we state some things about math. We can even do a math
  expression here, like so:
  \[ 1 + 1 \in \{1,2,3\} \]
 \end{thm}
 \begin{proof}
   This sentence is a proof. As you can see, it's long and complicated.
 \end{proof}
 \begin{dfn}
  A compiler is an algorithm that transforms one formal grammar into another.
  \footnote{We can even have a footnote in the Definition environment.}
 \end{dfn}
 \marginpar{\footnotesize\itshape{This diagram shows a rough sketch of a compiler's components.}}
 \[
 \begin{tikzcd}
     {\text{Source Code}} & {\text{Lexer}} \\
     & {\text{Parser}} \\
     & {\text{Compiler}} & {\text{Assembly}}
     \arrow[dashed, from=1-1, to=1-2]
     \arrow[from=1-2, to=2-2]
     \arrow[from=2-2, to=3-2]
     \arrow[dashed, from=3-2, to=3-3]
 \end{tikzcd}
 \]

%%%%%%%%%%%%%%%%%%%%%%%%%%%%%%%%%%%%%%%%%%%%%%%%%%
%-----% CHAPTER TWO %----------------------------%
 \chapter{The Chapter after the Beginning...}
 \epigraph{I really like these epigraph things that come before the chapters.} {\textsw{Reflections \& Translations}\\\textsc{Ethan Barry}}
 \lipsum[11]\MarginNote{\lipsum[11]}
 \begin{rem}
  This is our first remark. Isn't it remarkable?
 \end{rem}
 Here is a lovely commutative diagram made with
 the website \href{https://q.uiver.app}{q.uiver.app} which showcases some of the power of \LaTeX\,
 and the \verb|tikz| package.
 \begin{figure}[h!]
   % https://q.uiver.app/#q=WzAsOCxbMCwwLCJcXGRlbHRhIl0sWzIsMCwiXFx2YXJlcHNpbG9uIl0sWzAsMiwiXFx2YXJwaGkiXSxbMiwyLCJcXGdhbW1hIl0sWzEsMSwiXFxldGEiXSxbMywxLCJcXGlvdGEiXSxbMSwzLCJcXGthcHBhIl0sWzMsMywiXFx2YXJ0aGV0YSJdLFswLDRdLFswLDJdLFsyLDZdLFs0LDZdLFswLDFdLFs0LDVdLFsxLDVdLFs1LDddLFsxLDNdLFsyLDNdLFs2LDddLFszLDddXQ==
   \[
   \begin{tikzcd}
    \delta && \varepsilon \\
    & \eta && \iota \\
    \varphi && \gamma \\
    & \kappa && \vartheta
    \arrow[from=1-1, to=1-3]
    \arrow[from=1-1, to=2-2]
    \arrow[from=1-1, to=3-1]
    \arrow[from=1-3, to=2-4]
    \arrow[from=1-3, to=3-3]
    \arrow[from=2-2, to=2-4]
    \arrow[from=2-2, to=4-2]
    \arrow[from=2-4, to=4-4]
    \arrow[from=3-1, to=3-3]
    \arrow[from=3-1, to=4-2]
    \arrow[from=3-3, to=4-4]
    \arrow[from=4-2, to=4-4]
   \end{tikzcd}
   \]
   \caption{The caption goes here.}
 \end{figure}


 And we can follow this by some text and a huge equation. \lipsum[12]
 \begin{equation*}
    \mathrm{i}\,\hslash\frac{\partial}{\partial t} \Psi(\mathbf{r},t) = -\frac{\hslash^2}{2\,m}\nabla^2\Psi(\mathbf{r},t) + V(\mathbf{r})\Psi(\mathbf{r},t)
\end{equation*}

%%%%%%%%%%%%%%%%%%%%%%%%%%%%%%%%%%%%%%%%%%%%%%%%%%
%-----% CONCLUSION %-----------------------------%
\chapter*{Conclusion}
\lipsum[13]

Hopefully this provides you with a nice template to get started. Remember,
\begin{itemize}
  \item Change the global variables to use the existing design without much hassle,
  \item Decide what logo you want on the titlepage,
  \item Set up your margin notes, if you want them. (You should.)
  \item Don't forget to add something like \verb|biblatex-chicago| to manage your references.
  \item Start writing!
\end{itemize}

\vfill

\begin{center}
  {\large\swshape Soli Deo Gloria}
\end{center}

%%%%% END OF DOCUMENT %%%%%%%%%%%%%%%%%%%%%%%%%%%%
\end{document}
